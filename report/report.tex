\documentclass[10pt,twocolumn,letterpaper]{article}

\usepackage{iccv}
\usepackage{times}
\usepackage{epsfig}
\usepackage{graphicx}
\usepackage{amsmath}
\usepackage{amssymb}

% Include other packages here, before hyperref.

% If you comment hyperref and then uncomment it, you should delete
% egpaper.aux before re-running latex.  (Or just hit 'q' on the first latex
% run, let it finish, and you should be clear).
\usepackage[pagebackref=true,breaklinks=true,letterpaper=true,colorlinks,bookmarks=false]{hyperref}

 \iccvfinalcopy % *** Uncomment this line for the final submission

% Pages are numbered in submission mode, and unnumbered in camera-ready
\ificcvfinal\pagestyle{empty}\fi

\begin{document}

%%%%%%%%% TITLE
\title{UNC COMP 790 Project Proposal}

\author{Suneet Tipirneni\\
MSCV Student\\
{\tt\small suneet.tipirneni@knights.ucf.edu}}

\maketitle
% Remove page # from the first page of camera-ready.
\ificcvfinal\thispagestyle{empty}\fi


\section{Summary}

\section{Strengths}

Describe the problem that you are trying to solve in terms of the following:
\begin{itemize}
    \item Problem definition in terms of inputs/outputs.
    \item The motivation behind the problem.
    \item The goal of your project.
\end{itemize}


\section{Weaknesses}

Include a brief overview of the related work that covers:
\begin{itemize}
    \item Latest work in this area.
    \item How your approach relates to this prior work.
\end{itemize}

\section{Future Ideas}

Briefly describe the approach that you plan to use. This should include:

\begin{itemize}
    \item The high-level overview of your method.
    \item Reasons why you selected this method.
    \item More precise technical details related to the approach.
    \item References to the approaches that your method is built on.
    \item Technical novelty (if any) in relation to the prior work.
\end{itemize}

\begin{figure}[t]
\begin{center}
\fbox{\rule{0pt}{1.5in} \rule{0.7\linewidth}{0pt}}
   %\includegraphics[width=0.8\linewidth]{figure1.pdf}
\end{center}
   \caption{Optional figure illustrating the high-level idea of your approach or the problem that you are trying to solve.}
\label{fig:fig1}
\end{figure}

\section{Conclusion}

Provide details for your experimental setup:

\begin{itemize}
    \item The experiments that you plan to run.
    \item What you expect to discover from these experiments (for the proposal and milestone reports).
    \item Your results and the analysis of those results (for milestone and final reports).
    \item The datasets that you plan to use (along with a high-level overview of each dataset such as its size, annotations, when it was introduced, etc)
    \item The motivation why those datasets are suitable for your project.
\end{itemize}


%\begin{figure*}
%\begin{center}
%\fbox{\rule{0pt}{2in} \rule{.9\linewidth}{0pt}}
%%\includegraphics[width=0.8\linewidth]{figure2.pdf}
%\end{center}
%   \caption{Optional detailed illustration of your approach.}
%\label{fig:fig2}
%\end{figure*}

\begin{table}
\begin{center}
\begin{tabular}{c c}
\hline
Method & Accuracy \\
\hline
Theirs & 0.5 \\
Yours & 0.75\\
Ours & \bf 0.9 \\
\hline
\end{tabular}
\end{center}
\caption{Results for your milestone and final reports.}
\end{table}




{\small
\bibliographystyle{ieee_fullname}
\bibliography{egbib}
}

\end{document}
