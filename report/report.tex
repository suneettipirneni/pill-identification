\documentclass[10pt,twocolumn,letterpaper]{article}

\usepackage{iccv}
\usepackage{times}
\usepackage{epsfig}
\usepackage{graphicx}
\usepackage{amsmath}
\usepackage{amssymb}
\usepackage[numbers,compress]{natbib}

% Include other packages here, before hyperref.

% If you comment hyperref and then uncomment it, you should delete
% egpaper.aux before re-running latex.  (Or just hit 'q' on the first latex
% run, let it finish, and you should be clear).
\usepackage[pagebackref=true,breaklinks=true,letterpaper=true,colorlinks,bookmarks=false,colorlinks=true,allcolors=blue]{hyperref}


 \iccvfinalcopy % *** Uncomment this line for the final submission

% Pages are numbered in submission mode, and unnumbered in camera-ready
\ificcvfinal\pagestyle{empty}\fi

\begin{document}

%%%%%%%%% TITLE
\title{UCF CAP 5516 Project Proposal}

\author{Ronald Campos\\
MSCV Program\\
{\tt\small roncamposj@knights.ucf.edu}
\and
Suneet Tipirneni\\
MSCV Program\\
{\tt\small  suneet.tipirneni@knights.ucf.edu}
}

\maketitle
% Remove page # from the first page of camera-ready.
\ificcvfinal\thispagestyle{empty}\fi


\begin{abstract}
    As healthcare becomes more democratized, the need to validate potential harmful substances has become more and more important. This is especially true when it comes to medications. We want to propose an alternative way to identify pills by developing
    a model for identifying pills that relies on the robust and modern transformer architecture.
\end{abstract}


\section{Problem Definition}
Given an input of an image, our goal is to identity and provide a text output for pills/medications in the image. Having more tools to identify potentially hazardous substances can lead to safer environments especially in the pharmaceutical domain. The end goal of this project is to enable people with a non-pharmaceutical background the ability to identify pills/medications with ease using computer vision.

\section{Related Work}

Our related work is primarily comprised of the existing knowledge of vision transformers and the original paper
that performed pill identification using a CNN-like architecture.
\begin{itemize}
    \item An Image is Worth 16x16 Words: Transformers for Image Recognition at Scale~\cite{an_imageworth}
        \item A Low-Shot Fine-Grained Benchmark for Pill Identification~\cite{ePill}
\end{itemize}

\section{Method}
% Please add the following required packages to your document preamble:
% \usepackage{booktabs}

The pills in the dataset are structured in a way that they only ever appear once, making this 

\section{Dataset}
We will be using the ePillID dataset~\cite{ePill} for this work.  This dataset contains $13,000$ images and $4,902$ different pills.  There are $9,804$ different classes in the dataset, as there are two side to each pill.

\section{Experiments}
\begin{itemize}

    \item Different pre-processing techniques will be assessed. Some which will include various arrangements of image transformations, data normalization, and using different patch sizes.
    
    \item We will experiment with hyperparameter tuning in order to determine the most favorable set for this task.  Among these, we will test:

    \begin{itemize}
    
        \item A varying degree of transformer encoding blocks. 

        \item Using different activation functions.  We will consider activations like ReLU~\cite{relu2010} and GELU~\cite{gelu2016}, which are consistent with the standard architecture of Vision Transformers (ViTs)~\cite{an_imageworth}.  In addition to this, we will also look at the performance of classical approaches, such as Softmax~\cite{softmax}.

        
    \end{itemize}

    \item Our accuracies will then be compared with the CNN architecture used in~\cite{ePill}.
    % \item What you expect to discover from these experiments (for the proposal and milestone reports).
    % \item Your results and the analysis of those results (for milestone and final reports).
    % \item The datasets that you plan to use (along with a high-level overview of each dataset such as its size, annotations, when it was introduced, etc)
    % \item The motivation why those datasets are suitable for your project.
\end{itemize}

\section{Future Plans}

As shown above we currently achieve what we consider a sub-par accuracy for identification. As a result, we would like to look into methods that can improve the robustness of our apporach. One approach we are considering in the future is the usage of data augmentation. For this task we plan to use pre-existing solution called Cs$^2$. Cs$^2$ is an approach that uses generative models to create data augmentations to an already-existing dataset \cite{cs2}. By incorporating this into our current pipeline we hope that it can allow our model to train and see more variation which can hopefully result in overall better accuracies.  \par

In addition to more powerful data augmentations, we also would like to look into improvements to training the SWIN model. This could involve the un-freezing of specific layers in order for the model to better fit the inputs. We may also investigate other models that may be applicable. This is an area we are actively looking into at the time of writing this. \par 

% \section*{References}

% You should also include references to previous work.



%\begin{figure*}
%\begin{center}
%\fbox{\rule{0pt}{2in} \rule{.9\linewidth}{0pt}}
%%\includegraphics[width=0.8\linewidth]{figure2.pdf}
%\end{center}
%   \caption{Optional detailed illustration of your approach.}
%\label{fig:fig2}
%\end{figure*}

% \begin{table}
% \begin{center}
% \begin{tabular}{c c}
% \hline
% Method & Accuracy \\
% \hline
% Theirs & 0.5 \\
% Yours & 0.75\\
% Ours & \bf 0.9 \\
% \hline
% \end{tabular}
% \end{center}
% \caption{Results for your milestone and final reports.}
% \end{table}




{\small
\bibliographystyle{unsrt}
\bibliography{sample}
}

\end{document}
